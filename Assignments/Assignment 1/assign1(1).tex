\iffalse

INSTRUCTIONS: (if this is not lecture1.tex, use the right file name)

  Clip out the ********* INSERT HERE ********* bits below and insert
appropriate TeX code.  Once you are done with your file, run

  ``latex lecture1.tex''

from a UNIX prompt.  If your LaTeX code is clean, the latex will exit
back to a prompt.  Once this is done, run

  ``dvips lecture1.dvi''

which should print your file to the nearest printer.  There will be
residual files called lecture1.log, lecture1.aux, and lecture1.dvi.
All these can be deleted, but do not delete lecture1.tex.
\fi
%
\documentclass[11pt]{article}
\usepackage{amsfonts}
\usepackage{amsmath}
\usepackage{latexsym}
\usepackage{hyperref}

\setlength{\oddsidemargin}{.25in}
\setlength{\evensidemargin}{.25in}
\setlength{\textwidth}{6in}
\setlength{\topmargin}{-0.4in}
\setlength{\textheight}{8.5in}

\newcommand{\handout}[5]{
   %\renewcommand{\thepage}{#1-\arabic{page}}
   \noindent
   \begin{center}
   \framebox{
      \vbox{
    \hbox to 5.78in { {\bf Data Structures and Algorithms} \hfill #2 }
       \vspace{4mm}
       \hbox to 5.78in { {\Large \hfill #5  \hfill} }
       \vspace{2mm}
       \hbox to 5.78in { {\it #3 \hfill #4} }
      }
   }
   \end{center}
   \vspace*{4mm}
}

\newcommand{\lecture}[3]{\handout{L#1}{#2}{}{}{#1}}

\def\squarebox#1{\hbox to #1{\hfill\vbox to #1{\vfill}}}
\def\qed{\hspace*{\fill}
        \vbox{\hrule\hbox{\vrule\squarebox{.667em}\vrule}\hrule}}
\newenvironment{solution}{\begin{trivlist}\item[]{\bf Solution:}}
                      {\qed \end{trivlist}}
\newenvironment{solsketch}{\begin{trivlist}\item[]{\bf Solution Sketch:}}
                      {\qed \end{trivlist}}
\newenvironment{proof}{\begin{trivlist}\item[]{\bf Proof:}}
                      {\qed \end{trivlist}}

\newtheorem{theorem}{Theorem}
\newtheorem{corollary}[theorem]{Corollary}
\newtheorem{lemma}[theorem]{Lemma}
\newtheorem{observation}[theorem]{Observation}
\newtheorem{remark}[theorem]{Remark}
\newtheorem{proposition}[theorem]{Proposition}
\newtheorem{definition}[theorem]{Definition}
\newtheorem{Assertion}[theorem]{Assertion}
\newtheorem{fact}[theorem]{Fact}
\newtheorem{hypothesis}[theorem]{Hypothesis}
%\newtheorem{observation}[theorem]{Observation}
%\newtheorem{proposition}[theorem]{Proposition}
\newtheorem{claim}[theorem]{Claim}
\newtheorem{assumption}[theorem]{Assumption}

%Put more macros here, as needed.
\newcommand{\al}{\alpha}
\newcommand{\Z}{\mathbb Z}
\newcommand{\jac}[2]{\left(\frac{#1}{#2}\right)}
\newcommand{\set}[1]{\{#1\}}

\def\ppt{{\sf PPT}}
\def\poly{{\sf poly}}
\def\negl{{\sf negl}}
\def\owf{{\sf OWF}}
\def\owp{{\sf OWP}}
\def\tdp{{\sf TDP}}
\def\prg{{\sf PRG}}
\def\prf{{\sf PRF}}

%end of macros
\begin{document}

\lecture{COSC  336 		Assignment 1}{}{}

\textbf{Instructions.}
\begin{enumerate}
\item Due  Feb 16. 
\item This is a team assignment. Work in teams of 2-3 students.  Submit on Blackboard one assignment per team, with the names of all students making the team. 
\item Your programs must be written in Java.
\item Write your programs neatly - imagine yourself grading your program and see if it is easy to read and understand. 

Comment your programs reasonably: there is no need to comment lines like "i++" but do include brief comments describing the main purpose of a specific block of lines.
\item  You will submit on \textbf{Blackboard} two files.  

The \textbf{first file} is a pdf file (produced ideally with latex and Overleaf) and it will contain the following:
\begin{enumerate}
\item The solution to the Exercise.
\item   A short description of your algorithm for the Programming Task, where you explain the dynamic programing approach (see the sketch of the \textbf{Algorithm} below).  More precisely, you need to indicate how you  compute $d[0]$ (this is the initialization step), and how you compute for every $i \geq 1$, the value of $d[i]$ using the values of some of the previous $d[j]$'s, for $j < i$). 
\item   A table with the results your program gives  for the three data sets given below. 
\item   The java code (so that the grader can make observations).
\end{enumerate}


The \textbf{second file} is the .java file containing the java source code, so that the grader can run your program. 


\end{enumerate}
\newpage

\textbf{Exercise}

Consider the following three program fragments (a), (b), and (c). 
\begin{itemize}
\item[(a)]
\begin{verbatim}
sum = 0;
for (int i = 0; i< n ; i++) {
    	   sum++;
}
\end{verbatim}

\item[(b)]
\begin{verbatim}
sum = 0;
for (int i = 0; i< 2*n ; i++) {
    	   sum++;
}
\end{verbatim}

\item[(c)]
\begin{verbatim}
sum = 0;  i=n*n;
while (i > 1) {
    	   sum++;   
    	   i= i/2;
}
\end{verbatim}


\end{itemize}

We denote by $T_a(n), T_b(n), T_c(n)$ the running time of the three fragments.

\begin{enumerate}
\item  Give $\Theta$ evaluations for   $T_a(n), T_b(n), T_c(n)$.
\item Is  $T_b(n)  = O(T_a(n))$ ? Answer YES or NO and justify your answer.
\item  Is $T_c(n) = \Theta (T_a(n))$ ?  Answer YES or NO and justify your answer.
 

\end{enumerate}

\newpage
\textbf{Programming Task}. 

You will write a program that computes the length of a longest increasing subsequence of a sequence of integers. 

Formally, an increasing subsequence of the sequence $a_1, a_2, \ldots, a_n$ of length $k$ is given by $k$ indices $1 \leq i_1 < i_2 < \ldots < i_k \leq n$ such that $a_{i_1} < a_{i_2} < \ldots < a_{i_k}$. So the goal is to find the largest $k$ for which there exists an increasing subsequence of the input sequence of length $k$. Note: There is one major difference from the problem with \emph{max contiguos subsequence sum} which we discussed in class, namely in this problem  the subsequence is \textbf{not contiguous}, meaning that the numbers in the subsequence do not have to be in consecutive positions.

For example, if the input sequence is $10, 9, 2, 5, 3, 101, 7, 18$ then a longest increasing subsequence is $2,5,7, 18$, which has length $4$ (there is another increasing subsequence, namely $2,3,7,18$, also of length $4$). Therefore your program should return $4$ because there is no increasing subsequence of length $5$ or larger.

Your program will read the initial sequence which is entered by the user, and will print the length of a longest subsequence.  As a bonus, you may want your program to also print one longest increasing subsequence.
\smallskip

\textbf{Algorithm}  You will implement an algorithm using the dynamic programming paradigm, which is similar to Algorithm 3 for \emph{max contiguous subsequence sum} that we discussed in our meeting
 (see  Notes1-Intro on Blackboard).

 Suppose  the initial sequence is $a_0, a_1, \ldots, a_{n-1}$. Then,  you can calculate  in order, one by one,  the elements of an array $d[0], \ldots, d[n-1]$,  in which $d[i]$ is the length of the longest increasing subsequence whose last term is $a_i$. 
Think how to calculate $d[0]$ and then how to calculate $d[i]$ as a function of the previous entries $d[1],  \dots, d[i-1]$ and the sequence $a[]$.  % In the program, make a comment regarding how $d[i]$ is calculated from the previous $d[]$ values.
\medskip

\textbf{Example:}

Input: $10,9,2,5,3,101,7, 18$.
Output: $4$,  or for the bonus solution $4, (2,5,7,18)$.
\smallskip

Test your program on the following sequences and insert to the first file that you submit  screenshots with the computer screen showing the results for each sequence:

\begin{itemize}
\item $10,9,2,5,3,101, 7, 18$
\item 186, 359, 274,  927,  890,  520,  571,  310,  916,  798,  732,  23, 196, 579, 

426,188,  524,  991,   91,  150,  117,  565,  993,  615,   48, 811,  594,  303,  191,  

505,  724,  818,  536,  416,  179,  485 , 334  , 74,  998,  100,  197,  768,  421,  

114,  739,  636,  356,  908 , 477,  656
\item 318 ,  536  , 390  , 598  , 602 ,  408  , 254  , 868 ,  379  , 565  ,  206  ,  619  ,  936  ,  195 ,  

 123  ,  314  ,  729 ,  608  , 148 ,  540,   256 ,  768 ,  404  ,  190  ,  559 ,  1000 ,   482  ,  141 ,  26,   
 
  230  ,  550  ,  881  ,  759  ,  122 ,   878,    350,    756,     82,    562,    897,    508,    853,    317 ,   
  
  380 ,   807 ,    23 ,   506  ,   98 ,   757 ,   247
\end{itemize}









\end{document}
