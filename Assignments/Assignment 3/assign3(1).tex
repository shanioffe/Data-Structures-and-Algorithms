\iffalse

INSTRUCTIONS: (if this is not lecture1.tex, use the right file name)

  Clip out the ********* INSERT HERE ********* bits below and insert
appropriate TeX code.  Once you are done with your file, run

  ``latex lecture1.tex''

from a UNIX prompt.  If your LaTeX code is clean, the latex will exit
back to a prompt.  Once this is done, run

  ``dvips lecture1.dvi''

which should print your file to the nearest printer.  There will be
residual files called lecture1.log, lecture1.aux, and lecture1.dvi.
All these can be deleted, but do not delete lecture1.tex.
\fi
%
\documentclass[11pt]{article}
\usepackage{amsfonts}
\usepackage{amsmath}
\usepackage{latexsym}
\usepackage{hyperref}

\hypersetup{
    colorlinks=true,
    linkcolor=blue,
    filecolor=magenta,      
    urlcolor=cyan,
}
 
\urlstyle{same}

\setlength{\oddsidemargin}{.25in}
\setlength{\evensidemargin}{.25in}
\setlength{\textwidth}{6in}
\setlength{\topmargin}{-0.4in}
\setlength{\textheight}{8.5in}

\newcommand{\handout}[5]{
   %\renewcommand{\thepage}{#1-\arabic{page}}
   \noindent
   \begin{center}
   \framebox{
      \vbox{
    \hbox to 5.78in { {\bf Data Structures and Algorithms} \hfill #2 }
       \vspace{4mm}
       \hbox to 5.78in { {\Large \hfill #5  \hfill} }
       \vspace{2mm}
       \hbox to 5.78in { {\it #3 \hfill #4} }
      }
   }
   \end{center}
   \vspace*{4mm}
}

\newcommand{\lecture}[3]{\handout{L#1}{#2}{}{}{#1}}

\def\squarebox#1{\hbox to #1{\hfill\vbox to #1{\vfill}}}
\def\qed{\hspace*{\fill}
        \vbox{\hrule\hbox{\vrule\squarebox{.667em}\vrule}\hrule}}
\newenvironment{solution}{\begin{trivlist}\item[]{\bf Solution:}}
                      {\qed \end{trivlist}}
\newenvironment{solsketch}{\begin{trivlist}\item[]{\bf Solution Sketch:}}
                      {\qed \end{trivlist}}
\newenvironment{proof}{\begin{trivlist}\item[]{\bf Proof:}}
                      {\qed \end{trivlist}}

\newtheorem{theorem}{Theorem}
\newtheorem{corollary}[theorem]{Corollary}
\newtheorem{lemma}[theorem]{Lemma}
\newtheorem{observation}[theorem]{Observation}
\newtheorem{remark}[theorem]{Remark}
\newtheorem{proposition}[theorem]{Proposition}
\newtheorem{definition}[theorem]{Definition}
\newtheorem{Assertion}[theorem]{Assertion}
\newtheorem{fact}[theorem]{Fact}
\newtheorem{hypothesis}[theorem]{Hypothesis}
%\newtheorem{observation}[theorem]{Observation}
%\newtheorem{proposition}[theorem]{Proposition}
\newtheorem{claim}[theorem]{Claim}
\newtheorem{assumption}[theorem]{Assumption}

%Put more macros here, as needed.
\newcommand{\al}{\alpha}
\newcommand{\Z}{\mathbb Z}
\newcommand{\jac}[2]{\left(\frac{#1}{#2}\right)}
\newcommand{\set}[1]{\{#1\}}

\def\ppt{{\sf PPT}}
\def\poly{{\sf poly}}
\def\negl{{\sf negl}}
\def\owf{{\sf OWF}}
\def\owp{{\sf OWP}}
\def\tdp{{\sf TDP}}
\def\prg{{\sf PRG}}
\def\prf{{\sf PRF}}

%end of macros
\begin{document}

\lecture{COSC  336 		Assignment 3}{}{}

\textbf{Instructions.}
\begin{enumerate}
\item Due  date and time: see Blackboard.
\item This is a team assignment. Work in teams of 2-3 students.  Submit one assignment per team, with the names of all students making the team.
\item Your programs must be written in Java.
\item Write your programs neatly - imagine yourself grading your program and see if it is easy to read and understand. 
At the very beginning present your algorithm in plain English or in pseudo-code (or both).
Comment your programs reasonably: there is no need to comment lines like "i++" but do include brief comments describing the main purpose of a specific block of lines.

\item  You will submit on \textbf{Blackboard} two files.  

The \textbf{first file} is a pdf file (produced ideally with latex and Overleaf) and it will contain the following:
\begin{enumerate}
\item The solutions to the questions in Exercises.
\item   A short description of your algorithm for the programming task. Focus on how you have modified MERGE 
\item   A table with the results your program gives  for the three data sets given below. 
\item   The java code (so that the grader can make observations).
\end{enumerate}


The \textbf{second file} is the .java file containing the java source code, so that the grader can run your program. 

For editing the pdf file, I recommend that you use Latex, see the template files  posted on Blackboard:
 
           assignment-template.tex 	and  assignment-template.pdf

\end{enumerate}
\newpage


\textbf{Exercise 1.}  Analyze the following recurrences using the method that is indicated. In case you use the Master Theorem, state what the corresponding values of $a$, $b$, and $f(n)$ are and how
you determined which case of the theorem applies. 

\begin{itemize}
\item  $T(n) = 3 T(\frac{n}{4}) + 3$. Use the Master Theorem to find a $\Theta()$ evaluation, or say "Master Theorem cannot be used", if this is the case.
\item  $T(n) = 2 T(\frac{n}{2}) + 3n$. Use the Master Theorem to find a $\Theta()$ evaluation, or say "Master Theorem cannot be used", if this is the case.
\item  $T(n) = 9 T(\frac{n}{3}) + n^2 \log n $. Use the Master Theorem to find a $\Theta()$ evaluation, or say "Master Theorem cannot be used", if this is the case.

\end{itemize}
\bigskip

\textbf{Exercise 2.}
\begin{itemize}
\item $T(n) = 2T(n-1) + 1$, $T(0)=1$.  Use the iteration method to find a $\Theta()$ evaluation for $T(n)$.
\item $T(n) = T(n-1) + 1$,  $T(0)=1$.  Use the iteration method to find a $\Theta()$ evaluation for $T(n)$.
\item Give a  $\Theta( \cdot)$ evaluation for the runtime of the following code:
\begin{verbatim}
 i= n
 while(i >=1) {
    for (j=1;  j <=n;  j++)
        x=x+1
    i = i/2
}    
\end{verbatim}
\item Give a  $\Theta( \cdot)$ evaluation for the runtime of the following code:
\begin{verbatim}
 i= n
 while(i >=1) {
    for (j=1;  j <=i;  j++)
        x=x+1
    i = i/2
}
\end{verbatim}
\end{itemize}
\newpage
\textbf{Programming Task.}

The input is an array $a_1, a_2, \ldots, a_n$ of numbers.  A  \emph{*-pair} is a pair $(a_i, a_j)$  so that $1 \le i < j \le n$ and $a_i < a_j$. The task is to count the number of *-pairs
in the array.

For example, for $n = 5$ and input sequence $7, 3, 8, 1, 5$, we have the following *-pairs: $(7,8),  (3, 8),  (3, 5),  (1,5)$.  So, there are $4$ *-pairs. 
Design an $O(n \log n)$ algorithm which computes the number of *-pairs for a
given input sequence and implement your program in Java.

It's very easy to come up with an algorithm with run time  $\Theta(n^2)$ (just compare all pairs), but such an algorithm will not get any credit.

The idea of the $O(n \log n)$  algorithm is to modify MERGE-SORT (from page 34 in the textbook)  so that in addition to sorting the array it also counts the number of *-pairs. Thus, your modified MERGE-SORT (A, p, r) will sort the segment of the array $A[p..r]$  (like in the textbook), but in addition to that will return the number of *-pairs in this segment.  The main modification is in the MERGE  procedure, see textbook page 31. In the final for loop, you will also count the *-pairs in which the first component is in $L$, and the second component is in $R$. Keep in mind that $L$ and $R$ are both sorted and therefore if $L[i] < R[j]$, then all pairs $(L[i], R[j]),  (L[i], R[j+1]), \ldots, (L[i], R[r])$ are *-pairs. So with a single comparison you can add the number of these pairs to the counter of *-pairs. Using this observation the modified MERGE can be implemented in $O(n)$ (the same runtime as the standard MERGE).
\medskip




Test your program and report in a table the results for the following data sets.

Data set 1:  7,3,8,1,5

Data set 2:  The  numbers from the file input-3.4, available on Blackboard. The first line of the line has the number of elements (which is 1000), and the next line has the elements.

Data set 3:  The  numbers from the file input-3.5, available on Blackboard. The first line of the line has the number of elements (which is 10000), and the next line has the elements.

 

 



















 
  
\bigskip


NOTE 2:  You can find an example of what I mean by  ``Describe an algorithm ..." at

\url{https://www.geeksforgeeks.org/find-the-element-that-appears-once-in-a-sorted-array/}

%\url{https://www.geeksforgeeks.org/cut-all-the-rods-with-some-length-such-that-the-sum-of-cut-off-length-is-maximized/}
%\medskip

%See the text that starts with \textbf{Approach}.

\end{document}
